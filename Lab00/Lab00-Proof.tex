\documentclass[12pt,a4paper]{article}
\usepackage{ctex}
\usepackage{amsmath,amscd,amsbsy,amssymb,latexsym,url,bm,amsthm}
\usepackage{epsfig,graphicx,subfigure}
\usepackage{enumitem,balance}
\usepackage{wrapfig}
\usepackage{mathrsfs,euscript}
\usepackage[usenames]{xcolor}
\usepackage{hyperref}
\usepackage[vlined,ruled,linesnumbered]{algorithm2e}
\hypersetup{colorlinks=true,linkcolor=black}

\newtheorem{theorem}{Theorem}
\newtheorem{lemma}[theorem]{Lemma}
\newtheorem{proposition}[theorem]{Proposition}
\newtheorem{corollary}[theorem]{Corollary}
\newtheorem{exercise}{Exercise}
\newtheorem*{solution}{Solution}
\newtheorem{definition}{Definition}
\theoremstyle{definition}

\renewcommand{\thefootnote}{\fnsymbol{footnote}}

\newcommand{\postscript}[2]
 {\setlength{\epsfxsize}{#2\hsize}
  \centerline{\epsfbox{#1}}}

\renewcommand{\baselinestretch}{1.0}

\setlength{\oddsidemargin}{-0.365in}
\setlength{\evensidemargin}{-0.365in}
\setlength{\topmargin}{-0.3in}
\setlength{\headheight}{0in}
\setlength{\headsep}{0in}
\setlength{\textheight}{10.1in}
\setlength{\textwidth}{7in}
\makeatletter \renewenvironment{proof}[1][Proof] {\par\pushQED{\qed}\normalfont\topsep6\p@\@plus6\p@\relax\trivlist\item[\hskip\labelsep\bfseries#1\@addpunct{.}]\ignorespaces}{\popQED\endtrivlist\@endpefalse} \makeatother
\makeatletter
\renewenvironment{solution}[1][Solution] {\par\pushQED{\qed}\normalfont\topsep6\p@\@plus6\p@\relax\trivlist\item[\hskip\labelsep\bfseries#1\@addpunct{.}]\ignorespaces}{\popQED\endtrivlist\@endpefalse} \makeatother

\begin{document}
\noindent

%========================================================================
\noindent\framebox[\linewidth]{\shortstack[c]{
\Large{\textbf{Lab00-Proof}}\vspace{1mm}\\
CS214-Algorithm and Complexity, Xiaofeng Gao, Spring 2020.}}
\begin{center}
\footnotesize{\color{red}$*$ If there is any problem, please contact TA Yiming Liu.}

% Please write down your name, student id and email.
\footnotesize{\color{blue}$*$ Name: Hanzhang Yang  \quad Student ID:518030910022 \quad Email: linqinluli@sjtu.edu.cn}
\end{center}

\begin{enumerate}
    \item
    Prove that for any integer $n>2$, there is a prime $p$ satisfying $n<p<n!$. {\color{blue}(Hint: consider a prime factor $p$ of $n!-1$ and prove by contradiction)}
    \begin{proof}
       
        Assume  $n>2$ and there doesn't exist a prime p satisfying $n<p<n!$

       Then there must exist a prime factor of $n!-1$, $p$, and $p\le n!-1 <n!$ 
       
       Assume $p\le n$,so $p\mid n!$ and $p\mid n!-1$, then $p\mid 1$

       It's impossible. So $p>n$.

       Therefore there exists such a prime $p$ satisfying $n<p<n!$, which contradicts the assumption that there doesn't exist such a prime $p$.

   \end{proof}

    \item
    Use the minimal counterexample principle to prove that for any integer $n>17$, there exist integers $i_n\ge 0$ and $j_n\ge 0$, such that $n = i_n \times 4 + j_n \times 7$.
    \begin{proof}
        Define $P(n)$ be the statement that ``there exist integers $i_n\ge 0$ and $j_n\ge0$, such that $n=i_n\times 4 +j_n\times 7$''.
   
        If $P(n)$ is not true for every $n>17$, then there are values of $n$ for which $P(n)$ is false, and there must be a smallest such value, say $n = k$.

        Since we have $k>17$, and $k-1>16$.

        Since $k$ is the smallest value for which $P(k)$ is false, $P(k-1)$ is true.

        Thus $k-1=i_n \times 4+j_n \times 7$.

        If $j_n=0$, so $i_n>4$ ,
        
        However we have $$ k=k-1-4\times 5+7\times 3=(i_n-5)\times 4+(j_n+3)\times 7$$

        If $j_n\neq0$

        However we have $$ k=k-1+2\times 4-7=(i_n+2)\times 4+(j_n-1)\times 7$$

        We have derived a contradiction, which allows us to conclude that our original assumption is false.
    
    \end{proof}

    \item
    Let $P=\{p_1, p_2, \cdots\}$ the set of all primes. Suppose that $\{p_i\}$ is monotonically  increasing, i.e., $p_1=2$, $p_2=3$, $p_3=5$, $\cdots$. Please prove: $p_n<2^{2^n}$. {\color{blue}(Hint: $p_i \nmid (1+\prod_{j=1}^n p_j), i=1,2,\cdots,n$.)}
    \begin{proof}
        Define $P(n)$ be the statement that ``$p_n<2^{2^n}$''.

        \textbf{Basis step.} $P(1)$ is true

        \textbf{Induction hypothesis.} For $k\ge1$ and $1\le n\le k$,$P(n)$ is true.

        \textbf{Proof of induction step.} Let's prove $P(k+1)$.

        If $p_{k+1}\ge2^{2^{k+1}}$, there were only $k$ prime in $2^{2^{k+1}}$.
        
        So any composite numbers own the factor in $p_1,p_2,\cdots,p_k$

        However $p_i \nmid (1+\prod_{j=1}^n p_j), i=1,2,\cdots,n$

        since $\prod_{j=1}^k p_j+1=2^{2^{k+1}-2}+1<2^{2^{k+1}}$, We have derived a contradiction.

        So $p_{k+1}<2^{2^{k+1}}$

        Thus by induction hypothesis,$P(k+1)$ is true.

        Finally we can conclude that $p_n<2^{2^n}$.

        \end{proof}

    \item
    Prove that a plane divided by $n$ lines can be colored with only $2$ colors, and the adjacent regions have different colors.
    \begin{proof}
        Define $P(n)$ be the statement that ``a  plane  divided  by $n$ lines  can  be  colored  with  only  2  colors,  and  the  adjacentregions have different colors".
        
        \textbf{Basis step.} $P(1)$ is true.

        \textbf{Induction hypothesis.} For $k>1$, $P(k)$ is true.

        \textbf{Proof of induction step.} Let's prove $P(k+1)$.

        After drawing the $n$ line, we divide the plane into two sides. 
        Now, let's think about two adjacent regions $R_1$ and $R_2$. Change all of the colors into the different color in one side which includes $R_1$.

        If they are on the same side, their color is different since $P(k)$ is true.

        If they are on the different side, the color of the region including $R_1$ has been changed.
        So their color are different. $P(k+1)$ is true.

        Finally we can conclude that $P(n)$ is true.

    \end{proof}

\end{enumerate}

\vspace{20pt}

\textbf{Remark:} You need to include your .pdf and .tex files in your uploaded .rar or .zip file.

%========================================================================
\end{document}
