\documentclass[12pt,a4paper]{article}
\usepackage{ctex}
\usepackage{amsmath,amscd,amsbsy,amssymb,latexsym,url,bm,amsthm}
\usepackage{epsfig,graphicx,subfigure}
\usepackage{enumitem,balance}
\usepackage{wrapfig}
\usepackage{mathrsfs,euscript}
\usepackage[usenames]{xcolor}
\usepackage{hyperref}
\usepackage[vlined,ruled,linesnumbered]{algorithm2e}
\hypersetup{colorlinks=true,linkcolor=black}

\newtheorem{theorem}{Theorem}
\newtheorem{lemma}[theorem]{Lemma}
\newtheorem{proposition}[theorem]{Proposition}
\newtheorem{corollary}[theorem]{Corollary}
\newtheorem{exercise}{Exercise}
\newtheorem*{solution}{Solution}
\newtheorem{definition}{Definition}
\theoremstyle{definition}

\renewcommand{\thefootnote}{\fnsymbol{footnote}}

\newcommand{\postscript}[2]
 {\setlength{\epsfxsize}{#2\hsize}
  \centerline{\epsfbox{#1}}}

\renewcommand{\baselinestretch}{1.0}

\setlength{\oddsidemargin}{-0.365in}
\setlength{\evensidemargin}{-0.365in}
\setlength{\topmargin}{-0.3in}
\setlength{\headheight}{0in}
\setlength{\headsep}{0in}
\setlength{\textheight}{10.1in}
\setlength{\textwidth}{7in}
\makeatletter \renewenvironment{proof}[1][Proof] {\par\pushQED{\qed}\normalfont\topsep6\p@\@plus6\p@\relax\trivlist\item[\hskip\labelsep\bfseries#1\@addpunct{.}]\ignorespaces}{\popQED\endtrivlist\@endpefalse} \makeatother
\makeatletter
\renewenvironment{solution}[1][Solution] {\par\pushQED{\qed}\normalfont\topsep6\p@\@plus6\p@\relax\trivlist\item[\hskip\labelsep\bfseries#1\@addpunct{.}]\ignorespaces}{\popQED\endtrivlist\@endpefalse} \makeatother

\begin{document}

\noindent

%========================================================================
\noindent\framebox[\linewidth]{\shortstack[c]{
\Large{\textbf{Lab04-Matroid}}\vspace{1mm}\\
CS214-Algorithm and Complexity, Xiaofeng Gao, Spring 2020.}}
\begin{center}
\footnotesize{\color{red}$*$ If there is any problem, please contact TA Yiming Liu.}

% Please write down your name, student id and email.
\footnotesize{\color{blue}$*$ Name:Hanzhang Yang  \quad Student ID:518030910022 \quad Email: linqinluli@sjtu.edu.cn}
\end{center}

\begin{enumerate}
\item Give a directed graph $G=(V,E)$ whose edges have integer weights. Let $w(e)$ be the weight of edge $e\in E$. We are also given a constraint $f(u)\geq 0$ on the out-degree of each node $u\in V$. Our goal is to find a subset of edges with maximal weight, whose out-degree at any node is no greater than the constraint.
	\begin{enumerate}
	    \item Please define independent sets and prove that they form a matroid.
	    \item Write an optimal greedy algorithm based on Greedy-MAX in the form of \emph{pseudo code}.
	    \item Analyze the time complexity of your algorithm.
	\end{enumerate}

\textbf{Solution.}

(a)Define $(E,\mathcal{W})$ and $\mathcal{W}$ is the collection of all edge sets whose every connected node's out-degree is no greater than the constraint.

The following is the proof on the matroid.

\textbf{Hereditary.}

Assume $A\subset B,\ B\in \mathcal{W}$. Since $B\in\mathcal{W}$ and node's out-degree is no greater than $f(u)$, nodes in $A$ have lower out-degree than B.

Thus $A\in \mathcal{W}$.

\textbf{Exchange Property.}

Assume $A,B\in C$ and $|A|< |B|$.

case 1: There exists a edge $x\in B\backslash A$ and $x$ is not the out edge of node in A.

Thus $A\cup \{ x\}$'s every node's out-degree is still no greater than $f(u).$ $A\cup \{ x\} \in C$.

case 2: All edges in $B$ are the out edges of nodes  in A. Thus the graph consists of out edges both in A and B own the same nodes. Threr must be such a node which its out-degree in B is greater than that in A, since $|A|<|B|$.

Choose a edge x $x\in B\backslash A$ and x is the out edge of the node mentioned before.

Thus $A\in \mathcal{W}$.

(b)pseudo cide.

\begin{algorithm}[H]
	\KwIn{Graph $G=(V,E)$, weight of edges $W(i)$, constraint of nodes $f(i)$}
	\KwOut{a subset of edges with maximal weight, and every node's out-degree $\le$ $f(i)$}
	\BlankLine
	\caption{MaximalWeightGrapg}
	sort all edges by weight that $w(e_1)\ge w(e_2) \ge \cdots \ge w(e_n)$\;
	$R=\emptyset$\;
	$F[n]=\{ 0,0,\cdots,0\}$\;
	\For{$i=1$ to $n$}{
		\uIf{$F[m]< f(m)$}{
			$F[m]++$ $; \backslash \backslash e_i$ is the out edge of node $v_m$ \\
			add $e_i$ to $R$ \;
		}
		\Else{
			continue;
		}

	}
	\textbf{Output} $R$
\end{algorithm}

(c) The algorithm consists of a sort and a n-cycle.

The time complexity of the sort is $O(nlogn)$ and the cycle is $O(n)$.

Thus the time complexity of the algorithm is $O(nlogn).$


\item Let $X$, $Y$, $Z$ be three sets. We say two triples $\left(x_{1}, y_{1}, z_{1}\right)$ and $\left(x_{2}, y_{2}, z_{2}\right)$ in $X \times Y \times Z$ are \textit{disjoint} if $x_{1} \neq x_{2}$, $y_{1} \neq y_{2},$ and $z_{1} \neq z_{2}$. Consider the following problem:
    
    \begin{definition}[MAX-3DM] 
        Given three disjoint sets $X$, $Y$, $Z$ and a nonnegative weight function $c(\cdot)$ on all triples in $X \times Y \times Z$, \textbf{Maximum 3-Dimensional Matching} (MAX-3DM) is to find a collection $\mathcal{F}$ of disjoint triples with maximum total weight.
    \end{definition}

    \begin{enumerate}
    	\item Let $D = X \times Y \times Z$. Define independent sets for MAX-3DM.
    	\item Write a greedy algorithm based on Greedy-MAX in the form of \emph{pseudo code}. \label{Item-Greedy}
    	\item Give a counterexample to show that your Greedy-MAX algorithm in Q.~\ref{Item-Greedy} is not optimal.
    	\item Show that: $\max\limits_{F \subseteq D} \frac{v(F)}{u(F)} \leq 3$. {\color{blue}(Hint: you may need Theorem~\ref{Thm-Intersect} for this subquestion.)} 
    \end{enumerate}
    \begin{theorem} \label{Thm-Intersect}
        Suppose an independent system $(E, \mathcal{I})$ is the intersection of $k$ matroids $\left(E, \mathcal{I}_{i}\right)$, $1 \leq i \leq k$; that is, $\mathcal{I}=\bigcap_{i=1}^{k} \mathcal{I}_{i}$. Then $\max\limits_{F \subseteq E} \frac{v(F)}{u(F)} \leq k$, where $v(F)$ is the maximum size of independent subset in $F$ and $u(F)$ is the minimum size of maximal independent subset in $F$.
    \end{theorem}
 
	\textbf{Solution.}
	
	(a) Define $(S,\mathcal{C})$, and S is the set of all triples in $X\times Y \times Z$, $\mathcal{C}$ is the collection of all disjoint triples collection. The hereditary is easy to prove.

	(b) pseudo code 

	\begin{algorithm}[H]
		\KwIn{sets $X,\ Y,\ Z$ and weight c($\cdot$) on all triples in $X\times Y \times Z$}
		\KwOut{collection $\mathcal{F}$ of disjoint triples with maximum total weight }
		\BlankLine
		sort all triples by weight that $w(1,1,1)\ge w(1,1,2)\ge \cdots \ge w(x, y , z)$\;
		$\mathcal{F}=\emptyset$ \;
		\For{$i=1$ to $x$}{
			\For{$j=1$ to $y$}{
				\For{$k=1$ to $z$}{
					\uIf{$triple(i,j,k)$ is disjoint to all triples in $\mathcal{F}$}
						{add $triple(i,j,k)$ to $\mathcal{F}$ \;}
					\Else{
						continue\;
					}
				}
			}
		}
		\textbf{Output} $\mathcal{F}$
	\end{algorithm}

	(c) $X\{1,2\}\quad Y\{3,4\} \quad Z\{5,6\}$

	We give weight $8,7,7,7,7,7,7,1$ to $\{(1,3,5)\ (1,3,6)\ (1,4,5)\ (1,4,6)\ (2,3,5)\ (2,3,6)\ (2,4,5)\ (2,4,6)\}$

	From the algorithm we choose $\{(1,3,5)\ (2,4,6)\}$ as the final set and the weight is 9. However it's clear that $\{(1,3,6)\ (2,4,5)\}$ with weight 14 is optimal.

	(d) \textbf{Proof.}

	Assume an independent system $(S,\mathcal{X})$. $\mathcal{X}$ is the collection of all collections of triples in $X\times Y\times Z$ with disjoint \textbf{x}$\in X$.

	\textbf{Hereditary}

	Since $A\subset B,\ B\in \mathcal{X}$, it's easy to see that $A\in \mathcal{X}$.

	\textbf{Exchange Property}

	Since $|A|<|B|$, the values of $x$ in $B$ is more than in $A$. Thus there must exists a triple $(m,y,z)\in B$ and $m$ haven't exist in $A's$ triples before.
	$|A|\cup {(m,y,z)}\in \mathcal{X}$. 

	Therefore $(S,\mathcal{X})$ is a matroid.

	We can get $(S,\mathcal{Y})$ and $(S,\mathcal{Z})$ in the same way.

	Then $$\mathcal{X} \cap \mathcal{Y} \cap\mathcal{Z} =\mathcal{C},$$
	for $\mathcal{C}$ is the collection of all collections with disjoint $x,\ y, \ and \ z$.

	Finally we can draw the conclusion that $\max\limits_{F \subseteq D} \frac{v(F)}{u(F)} \leq 3$ by theorem 1.
	\item
	\textbf{Crowdsourcing} is the process of obtaining needed services, ideas, or content by soliciting contributions from a large group of people, especially an online community. Suppose you want to form a team to complete a crowdsourcing task, and there are $n$ individuals to choose from. Each person $p_i$ can contribute $v_i$ ($v_i > 0$) to the team, but he/she can only work with up to $c_i$ other people. Now it is up to you to choose a certain group of people and maximize their total contributions ($\sum_i{v_i}$).
	
	\begin{enumerate}
		\item Given $\text{OPT}(i, b, c)=$ maximum contributions when choosing from $\{p_1, p_2, \cdots, p_i\}$ with $b$ persons from $\{p_{i+1}, p_{i+2}, \cdots, p_n\}$ already on board and at most $c$ seats left before any of the existing team members gets uncomfortable. Describe the optimal substructure as we did in class and write a recurrence for $\text{OPT}(i, b, c)$.
		\item Design an algorithm to form your team using dynamic programming, in the form of \emph{pseudo code}.
        \item Analyze the time and space complexities of your design.
	\end{enumerate}
\end{enumerate}

\textbf{Solution.}

(a)When choosing the $i_{th}$ person, there are two cases.


\textbf{Case 2:} $OPT(i,b,c)=max\{ OPT(i-1,b,c),\ v_i+OPT(i-1,b+1,min\{ c_i-b,c\})\}$

\begin{equation}
    OPT(i,b,c)=
   \begin{cases}
   0&i=0\\
   OPT(i-1,b,c)& c=0\ or \ c_i<b\\
   max\{ OPT(i-1,b,c),\ v_i+OPT(i-1,b+1,min\{ c_i-b,c\})\}&else
   \end{cases}
  \end{equation}

  (b)\textbf{pseudo code:}

  \begin{algorithm}[H]
	\KwIn{$P\{(v_1,c_1),\ (v_2,c_2),\cdots,(v_n,c_n)\}$}
	\KwOut{a set of people}
	\BlankLine
	$A[n]=\{ 0,0,\cdots,0\};\backslash \backslash$ used to store the result, not in the recursion\\
	\caption{OPT(i,b,c)}
	\uIf{$i=0$}{
		return 0\;
	}
	\uElseIf{$c=0\ or \ c_i<b$}{
		return $OPT(i-1,b,c)$\;
	}
	\uElse{
		\uIf{$OPT(i-1,b,c)\le v_i+OPT(i-1,b+1,min\{ c_i-b,c\})$}{
			A[i]=1\;
			return $v_i+OPT(i-1,b+1,min\{ c_i-b,c\})$\;
		}
		\uElse{
			return $OPT(i-1,b,c)$\;
		}
		
	}
	\BlankLine
	$OPT(n,0,n);\backslash \backslash$ the begin of the recursion\\
	\textbf{Output} A\;

\end{algorithm}

(c) The deep of the recursion is $n$.

\textbf{Space Complexity:$O(n)$}

The space complexity is $O(n)$, since every recursion costs constant space.

\textbf{Time complexity:$O(n^3)$}

Consider the worst case, which chooses $n$ people. For the deep of the recursion is n, and in the $i_{th}$ recursion the time cost is $2^{i-1}.$

Thus the time complexity is $O(2^n)$.

And using a three-dimensional array to  store the result can make the time community become $O(n^3)$ and the space complexity become $O(n^3)$.
\vspace{20pt}

\textbf{Remark:} You need to include your .pdf and .tex files in your uploaded .rar or .zip file.

%========================================================================
\end{document}
